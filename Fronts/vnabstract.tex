%% ================= ABSTRACT IN VIETNAMSES ===================== %%
\begin{vnabstract}
\selectlanguage{vietnam} %% Must be presented here, or you will not be able to use Vietnamese in some pages in your thesis
	\indent Con người có thể nhận biết và mô tả thế giới xung quanh mình trong một khoảng thời gian rất ngắn (cỡ phần nghìn giây). Tuy nhiên, đây lại là một thách thức lớn đối với máy tính, ngay cả với những hệ thống thị giác máy tính tân tiến nhất. Việc tự động nhận diện và mô tả nội dung một hình ảnh dưới dạng những câu mô tả Tiếng Anh có cấu trúc tương đối hoàn chỉnh có nhiều ứng dụng trong thực tế.  Những hệ thống có khả năng kiểu như vậy có thể được sử dụng để xây dựng các ứng dụng trợ giúp người khiếm thị, giúp họ nhận biết được thế giới xung quanh nhằm di chuyển một cách dễ dàng hơn. Một ứng dụng thực tế hơn nữa là hỗ trợ các hệ thống tìm kiếm hình ảnh (image search engines). Thay vì tìm kiếm dựa trên đặc trưng mức thấp của hình ảnh (có thể phải mất nhiều thời gian trích chọn và không gian lưu trữ), máy tìm kiếm có thể tìm dựa trên những mô tả của hình ảnh ấy. Điều này giúp làm giảm thời gian tìm kiếm và cho những kết quả gần hơn với mong muốn truy vấn của người dùng.

	Bài toán tự động sinh mô tả hình ảnh là một bài toán có tính chất kết nối hai lĩnh vực thị giác máy tính và dịch máy. Trong phạm vi đồ án này, tôi tìm hiểu một phương pháp tiếp cận bài toán này sử dụng mạng nơ-ron. Ý tưởng của hướng tiếp cận này là dùng mạng nơ-ron chập (\gls{cnn}) để học một véc-tơ đặc trưng của hình ảnh có kích thước cố định, sau đó véc-tơ này được sử dụng làm đầu vào cho mạng nơ-ron lặp (\gls{rnn}) để sinh ra câu mô tả hình ảnh. Mô hình này học để tối đa xác suất sinh ra từ $S_t$ khi đã có véc-tơ đặc trưng hình ảnh $I$ và các từ $S_0$ cho đến $S_{t-1}$.

	Đồ án khảo sát và đánh giá ảnh hướng của những yếu tố khác nhau đến hiệu quả của mô hình: mô-đun trích chọn đặc trưng hình ảnh (\gls{cnn}), kích thước của \gls{rnn}, những phương pháp tối ưu khác nhau, v.v. nhằm tìm ra những lựa chọn thích hợp nhất cho mô hình sinh mô tả hình ảnh, cân bằng giữa hiệu năng và tính khả thi khi triển khai mô hình lên các hệ thống có hạ tầng phần cứng ở mức trung bình. Tôi đã tiến hành thử nghiệm các tham số của mô hình trên tập dữ liệu MSCOCO và đánh giá mô hình dựa trên những phương pháp đánh giá chuẩn được sử dụng cho các bài toán sinh mô tả hình ảnh và dịch máy. Trên cơ sở những đánh giá ấy, tôi đề xuất những bộ tham số thích hợp của mô hình cho những ứng dụng cụ thể.

\selectlanguage{english}
\end{vnabstract}